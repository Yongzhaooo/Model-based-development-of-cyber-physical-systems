\section{Problem 1}

\subsection{Prove}

Given the transformation matrix $^W\xi_B$, I renote it as:

\begin{equation}
    \begin{aligned}
        ^W\xi_B &=\begin{bmatrix} R & d \\ \boldsymbol{0} & 1 \end{bmatrix}\\
        R & = \begin{bmatrix} r11 & r12 & r13 \\ r21 & r22 & r23 \\r31 & r32 &r33 \end{bmatrix}\\
        d & = \begin{bmatrix} t1 \\ t2 \\t3 \end{bmatrix}
    \end{aligned}
\end{equation}

$R $is a rotation matrix and $d$ is a translation vector.

Since $R$ is an orthogonal matrix, the inverse of the rotation matrix $R$ is its transpose, 

\begin{equation}
    \begin{aligned}
        R^{-1} &= R^T\\
        RR^T&=R^TR= \boldsymbol{I}
    \end{aligned}
\end{equation}

Define a matrix $ H^-1 $ as :

\begin{equation}
    \begin{aligned}
        H^{-1} &= \begin{bmatrix} R^T & -R^Td \\ \boldsymbol{0} & 1 \end{bmatrix} 
    \end{aligned}
\end{equation}

And see the result of $ ^W\xi_B \times H^{-1} $:

\begin{equation}
    \begin{aligned}
        ^W\xi_B \times H^{-1} = \begin{bmatrix} R & d \\ \boldsymbol{0} & 1 \end{bmatrix} \begin{bmatrix} R^T & -R^Td \\ \boldsymbol{0} & 1 \end{bmatrix} = \begin{bmatrix} RR^T & -RR^Td+d\\\boldsymbol{0} & 1 \end{bmatrix}=\boldsymbol{\begin{bmatrix} 1 &0\\0&1 \end{bmatrix}}
    \end{aligned}
\end{equation}

which means:
\begin{equation}
    \begin{aligned}
        ^B{\xi}_{W} = H^{-1} = \begin{bmatrix}
            R^T & -R^T d \\
            \mathbf{0} & 1
            \end{bmatrix}
    \end{aligned}
\end{equation}

Q.E.D.

\subsection{Calculate}

I select my $ ^B{\xi}_{W} $ and $ p_W $ and calculate through Matlab:
\begin{lstlisting}
    W_H_B = [ 0 0 1 1;
    1 0 0 2;
    0 1 0 3;
    0 0 0 1 ];

    p_W = [ 2; 2; 1; 1 ]

    B_H_W = inv(W_H_B);
    p_B = B_H_W * p_W

\end{lstlisting}

Output result:

\begin{lstlisting}
    p_W = 4×1    
     2
     2
     1
     1


     p_B = 4×1    
     0
    -2
     1
     1

\end{lstlisting}

It show, the p point in $W$ coordinate has position as $[2\  2\  1]$, and p in $B$ coordinate has position as $[0\  -2\  1]$.

